\section{Agile Methoden und Open Source}
(Autor: Henn)\\

Agile Methoden sind zur Zeit voll im Trend. Die Fachzeitschriften sind voll davon und in Firmen
werden agile Vorgehensweisen immer häufiger umgesetzt. Doch wie sieht die Situation in Open Source
Projekten mit ganz anderen Umständen und Herausforderungen aus?

In diesem Abschnitt wird ein Blick in die Open Source Welt geworfen und
analysiert, inwiefern agile Techniken eingesetzt werden. Im Anschluss werden
Chancen und Probleme bei der Verwendung eines agilen Modells im Open Source
Bereich dargestellt.


\subsection{Open Source Projekte}
Bei der Recherche viel auf, dass es gar nicht so leicht ist, Open Source
Projekte zu finden, die von sich selbst behaupten, sie würden nach agilen
Vorgehensmodellen vorgehen. Allerdings finden sich auch in normalen Projekten
Techniken, die eindeutig aus der agilen Softwareentwicklung kommen. Im Folgenden
sollen einige Projekte als Stellvertreter für die verschiedenen Stufen der
Adaption mit Blick auf die Umsetzung agiler Techniken und Prinzipien betrachtet werden.

\subsubsection{TYPO3}
Die Entwickler des Open Source Content Management Systems TYPO3 haben sich dazu
entschieden, ihre nächste Version 5 (Phoenix) vollständig nach dem Scrum Vorgehensmodell zu
entwickeln. Anhand dieses Beispiels soll gezeigt werden, wie es möglich ist,
ein solches Vorgehensmodell im Open Source Bereich vollständig umzusetzen und
was es für Bedingungen gibt, damit dies erfolgreich geschehen kann.
\begin{figure}[h]
	\centering
	\includegraphics[width=1\textwidth]{images/typo3_Phoenix_logo.jpg}
	\caption{Logo TYPO3 5.0 Phoenix}
	\label{Logo-Phoenix}
\end{figure}

Ein Sprint dauert beim TYPO3 Phoenix Team 4 Wochen. Innerhalb dieses Sprints finden tägliche
Meetings statt und am Ende eines Sprints muss eine lauffähige Version bereit stehen. Abbildung 
\ref{magic-cycle} visualisiert diesen Zyklus. User Stories aus dem Product Backlog werden für
diesen Sprint ausgesucht und in einem Sprint Backlog festgehalten. Diese Stories werden dann
innerhalb der 2-4 Wochen implementiert.
\begin{figure}[h]
	\centering
	\includegraphics[width=1\textwidth]{images/typo3-magic-cycle.jpg}
	\caption{Magic Cycle - TYPO3 Scrum Implementierung}
	\label{magic-cycle}
\end{figure}


Als Product-Owner, der in Firmen in der Regel aus einer Person besteht, wurde für TYPO3
ebenfalls ein kleines Team bestehend aus drei Personen eingerichtet. Dies erscheint sinnvoll, da es
sich bei TYPO3 ja um eine Gemeinschaftsentwicklung handelt. Eine einzelne Person als Product-Owner
würde dem Community Gedanken entgegen wirken und  wäre darüber hinaus Person wahrscheinlich auch
noch überlastet, wenn sie plant, diese Aufgabe in ihrer Freizeit auszufüllen. Diese drei
Personen diskutieren über die Features, die innerhalb eines Sprints implementiert werden sollen
und priorisieren diese.
\newline Die Rolle des Scrum-Masters wird wiederum nur von einer Person besetzt. Auch dies ergibt
durchaus Sinn, da sich zwei Scrum-Master, oder sogar ein Team von Scrum-Masters eher gegenseitig bei
der Leitung von Gesprächen behindern würde.
\newline Das eigentliche Scrum-Team für die TYPO3 Entwicklung besteht aus 19 Entwicklern. Die alle
miteinander in Kontakt stehen und koordiniert werden müssen. Entscheidend für den Erfolg von Scrum
ist daher das Internet. Es bietet vielfältige Möglichkeiten, miteinander zu kommunizieren und große
räumliche Entfernungen zu überbrücken. So kommuniziert das Team:
\begin{itemize}
\item Sprint-Planung, Review und Retrospektive werden durch Meetings im Internet durchgeführt.
\item Es steht ein Jabber Chatroom bereit, in dem sich das Team kurzfristig abstimmen kann und
Fragen während der Entwicklung geklärt werden.
\item jeden Tag findet um 17:00 Uhr das Daily Scrum Meeting mittels Webex oder Skype statt. Danach
wird das Protokoll der Sitzung über die Mailing Liste verschickt, damit alle, die nicht teilnehmen
konnten, trotzdem auf dem neusten Stand sind.
\item Das Sprint-Backlog wird ebenfalls im Internet unter forge.typo3.org gepflegt.
\item TYPO3 Veranstaltungen werden so oft, wie möglich genutzt um persönlich miteinander sprechen zu
können
\item Nach jedem Sprint wird das Ergebnis sowohl als Nachricht, als auch als laufende
Demo-Installation bekanntgegeben
\end{itemize}
\begin{figure}[h]
	\centering
	\fbox{
		\includegraphics[width=1\textwidth]{images/forge-typo3-org.png}
	}
	\caption{forge.typo3.org}
	\label{forge}
\end{figure}
Der gesamte Fortschritt sowie die offizielle Dokumentation und Kommunikation wird über
die Web-Plattform forge.typo3.org dargestellt. Auf diese Weise ist der Entwicklungsprozess sehr
transparent und für die Community einsehbar. Dadurch verliert die Entwicklung nicht an Akzeptanz und
die freiwilligen Committer wissen, was gerade benötigt wird.

Auf der Website des Phoenix Teams wurden mehrere Bereiche eingerichtet:
\begin{description}
\item [Core Team] Hier befinden sich alle Informationen, die das Entwicklungs-Team zur Koordination
benötigt. der Bereich beinhaltet
\begin{itemize}
\item eine Übersicht mit Link zum Issue-Tracker und einer Liste der Team Mitglieder
\item Eine Auflistung aller bisher ausgeführten Aktivitäten, wie Wiki Änderungen, oder auch
Änderungen am Quellcode
\item Eine Roadmap für den aktuellen Sprint, mit Auflistung aller in diesem Sprint zu erledigenden
User Stories und den dazugehörigen Tasks
\item den Issue Tracker
\item das Sprint, sowie das Product Backlog
\item ein Gantt Chart
\item ein Kalender
\item  ein Wiki
\item das Repository mit dem Quellcode
\end{itemize}
Das Wiki beinhaltet alle wichtigen Informationen für Entwickler, sowie teilweise Erklärungen zum
Scrum Prozess und Richtlinien.  So findet sich dort zum Beispiel die Erklärung, was es mit der
''Definition of Done``  auf sich hat, die bei Scrum ein wichtiges Artefakt darstellt.
\item [Product Owner Team] Für das Product Owner Team gibt es ebenfalls wie beim Core Team eine
Übersicht mit den Mitgliedern, sowie eine kurze Beschreibung der Gründe, sich für ein Teams als
Product Owner zu entscheiden. Es finden sich ebenso Übersichten über Aktivitäten, Issues und
Termine. Außerdem wird ebenfalls ein Wiki gepflegt. Dieses Wiki behandelt  allerdings viel
stärker den Scrum Entwicklungsprozess, als das Wiki des Core Teams. So ist hier der eigentliche
Entwicklungsprozess für TYPO3 Phoenix beschrieben, sowie für alle, die Scrum noch nicht so intuitiv
leben ein kleines Cheat Sheet hinterlegt, das den Scrum Prozess nochmal auf einer Seite
übersichtlich zusammenfasst. Ebenso findet sich hier eine Liste mit Benutzerrollen für TYPO3. Im
Anschluss daran sind die Protokolle der einzelnen Scrum Meetings hinterlegt.
\item [Scrum Master] im Bereich des Scrum Masters gibt es lediglich eine Übersicht, einen Bereich
in dem die Aktivitäten aufgelistet werden, sowie Issue Tracker, Gantt Chart und ein Kalender. Da
die Rolle des Scrum Masters nur mit einer Person besetzt ist, muss hier auch nicht mehr Inhalt
bereit stehen.
\item [UI/UX Team] dieses Team beschäftigt sich mit dem User Interface, sowie der User Experience.
Es finden sich die Übersichten, über Issues und Aktivitäten, wie in den anderen Bereichen. 
\end{description}

Die TYPO3 Entwickler haben es geschafft, den gesamten Scrum Prozess mit minimalen Anpassungen
für ihr Projekt zu Adaptieren. Von entscheidendem Vorteil für die TYPO3 Entwickler, ist die
Tatsache, dass sie alle aus Deutschland kommen. So ist es deutlich einfacher, einen Termin, der auf
17:00 Uhr angesetzt ist, wahrzunehmen. Eine solche regelmäßige zeitliche Abstimmung würde bei
Entwicklern, die über die ganze Welt verstreut sind nicht funktionieren. Außerdem ist es so für die
Projektteilnehmer leichter, zu den TYPO3 Veranstaltungen zu fahren, da diese ebenfalls hauptsächlich
in Deutschland stattfinden. Auch hier wäre es für einen weit verstreutes Entwicklerteam
auch finanziell nicht möglich, sich so oft persönlich zu sehen. Ein weiterer Faktor ist sicherlich
die Größe des Projektteams. Bei nur 17 Mitgliedern ist es leichter möglich, alle über das Internet
zu koordinieren und dafür zu sorgen, dass auch wirklich alle gerade aktiv am Projekt
teilnehmen können.



\subsection{Vorteile agiler Methoden in Open Source Projekten}
  - Motivation von Entwicklern und Community\\
  - Rasches Umsetzen neuer Anforderungswünsche\\
  - ...
  
\subsection{Nachteile agiler Methoden in Open Source Projekten}
  - Zeit\\
  - Ort\\
  - ...



\section{Agile Methoden und Open Source}
(Autor: Henn)\\
Agile Methoden sind zur Zeit voll im Trend. Die Fachzeitschriften sind voll davon und in Firmen
werden agile Vorgehensweisen immer häufiger umgesetzt. Doch wie sieht die Situation in Open Source
Projekten mit ganz anderen Umständen und Herausforderungen aus?

In diesem Abschnitt wird ein Blick in die Open Source Welt geworfen und
analysiert, inwiefern agile Techniken eingesetzt werden. Im Anschluss werden
Chancen und Probleme bei der Verwendung eines agilen Modells im Open Source
Bereich dargestellt.


\subsection{Open Source Projekte}
Bei der Recherche viel auf, dass es gar nicht so leicht ist, Open Source
Projekte zu finden, die von sich selbst behaupten, sie würden nach agilen
Vorgehensmodellen vorgehen. Allerdings finden sich auch in normalen Projekten
Techniken, die eindeutig aus der agilen Softwareentwicklung kommen. Im Folgenden
sollen einige Projekte als Stellvertreter für die verschiedenen Stufen der
Adaption betrachtet werden.

\subsubsection{TYPO3}
Die Entwickler des Open Source Content Management Systems TYPO3 haben sich dazu
entschieden, ihre nächste Version vollständig nach dem Scrum Vorgehensmodell zu
entwickeln. Anhand dieses Beispiels soll gezeigt werden, wie es möglich ist,
ein solches Vorgehensmodell im Open Source Bereich vollständig umzusetzen und
was es für Bedingungen gibt, damit dies erfolgreich geschehen kann.


Als Product-Owner, der in Firmen in der Regel aus einer Person besteht, wurde für TYPO3
ebenfalls ein kleines Team bestehend aus drei Personen eingerichtet. Dies macht Sinn, da es sich bei
TYPO3 ja um eine Gemeinschaftsentwicklung handelt. Eine einzelne Person als Product-Owner würde da
keinen Sinn ergeben und dazu wäre diese Person wahrscheinlich auch noch überlastet, wenn sie diese
Aufgabe in ihrer Freizeit ausfüllen möchte. Diese drei Personen diskutieren über die Features, die
innerhalb eines Sprints implementiert werden sollen und priorisieren diese.
\newline Die Rolle des Scrum-Masters wird wiederum nur von einer Person besetzt. Auch dies macht
durchaus Sinn, da sich zwei Scrum-Master, oder sogar ein Team von Scrum-Masters eher gegenseitig bei
der Leitung von Gesprächen behindern würde.
\newline Das eigentliche Scrum-Team für die TYPO3 Entwicklung besteht aus 19 Entwicklern. Die alle
miteinander in Kontakt stehen und koordiniert werden müssen.

Entscheidend für den Erfolg von Scrum ist das Internet. Es bietet vielfältige Möglichkeiten,
miteinander zu kommunizieren und große räumliche Entfernungen zu überbrücken. So kommuniziert das
Team:
\begin{itemize}
\item Sprint-Planung, Review und Retrospektive werden durch Meetings im Internet durchgeführt.
\item Es steht ein Jabber Chatroom bereit, in dem sich das Team kurzfristig abstimmen kann und
Fragen währen der Entwicklung geklärt werden.
\item jeden Tag findet um 17:30 Uhr das Daily Scrum Meeting statt. Danach wird das Protokoll der
Sitzung über die Mailing Liste verschickt, damit alle, die nicht teilnehmen konnten, trotzdem auf
dem neusten Stand sind.
\item Das Sprint-Backlog wird ebenfalls im Internet unter forge.typo3.org gepflegt.
\item TYPO3 Veranstaltungen werden so oft, wie möglich genutzt um persönlich miteinander sprechen
zu können
\end{itemize}

Von entscheidendem Vorteil für die TYPO3 Entwickler, ist die Tatsache, dass sie alle aus
Deutschland kommen. So ist es deutlich leichter, einen Termin, der auf 17:30 Uhr angesetzt ist
wahrzunehmen. Eine solche regelmäßige zeitliche Abstimmung würde bei Entwicklern, die über die
ganze Welt verstreut sind nicht funktionieren. Außerdem ist es so für die Projektteilnehmer
leichter, zu den TYPO3 Veranstaltungen zu fahren, da diese ebenfalls hauptsächlich in Deutschland
stattfinden. Auch hier wäre es für einen weit verstreutes Entwicklerteam auch finanziell nicht
möglich, sich so oft persönlich zu sehen.


\subsection{Vorteile agiler Methoden in Open Source Projekten}
  - Motivation von Entwicklern und Community\\
  - Rasches Umsetzen neuer Anforderungswünsche\\
  - ...
  
\subsection{Nachteile agiler Methoden in Open Source Projekten}
  - Zeit\\
  - Ort\\
  - ...



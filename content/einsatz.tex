\section{Agile Methoden und Open Source}
(Autor: Henn)\\
Agile Methoden sind zur Zeit voll im Trend und in Firmen werden agile
Vorgehensweisen immer häufiger umgesetzt. Doch wie sieht die Situation in Open
Source Projekten mit ganz anderen Umständen aus?

In diesem Abschnitt wird ein Blick in die Open Source Welt geworfen und
analysiert, inwiefern agile Techniken eingesetzt werden. Im Anschluss werden
Chancen und Probleme bei der Verwendung eines agilen Modells im Open Source
Bereich dargestellt.


\subsection{Open Source Projekte}
Bei der Recherche viel auf, dass es gar nicht so leicht ist, Open Source
Projekte zu finden, die von sich selbst behaupten, sie würden nach agilen
Vorgehensmodellen vorgehen. Allerdings finden sich auch in normalen Projekten
Techniken, die eindeutig aus der agilen Softwareentwicklung kommen. Im Folgenden
sollen einige Projekte als Stellvertreter für die verschiedenen Stufen der
Adaption betrachtet werden.
\subsubsection{Typo3}
Die Entwickler des Open Source Content Management Systems Typo3 haben sich dazu
entschieden, ihre nächste Version vollständig nach dem Scrum Vorgehensmodell zu
entwickeln. 
\subsection{Vorteile agiler Methoden in Open Source Projekten}
  - Motivation von Entwicklern und Community\\
  - Rasches Umsetzen neuer Anforderungswünsche\\
  - ...
  
\subsection{Nachteile agiler Methoden in Open Source Projekten}
  - Zeit\\
  - Ort\\
  - ...



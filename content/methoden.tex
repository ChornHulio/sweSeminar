\section{Vorgehensmodelle}
(Autor: Dreher)
Wie in Kapitel \ref{sec:einfuehrung} beschrieben, basieren alle Vorgehensmodelle auf wenigen klaren Regeln. Die bedeutendsten Modelle werden in diesem Kapitel genauer erläutert.

\subsection{Extreme Programming}
Extreme Progamming (XP) ist eine der ältesten agilen Vorgehensmodelle und gibt Vorgaben für die Planung, das Management, den Entwurf, das Testmanagement und die Programmierung eines Softwareprojekts. Dadurch wird XP sehr mächtig, aber auch anspruchsvoll für alle Beteiligten. \cite{bib:wolfRoock} \cite{bib:xp}

\subsubsection{Planung}
Die Anforderungen an die Softwarelösung werden in sogenannten \emph{User Stories} festgehalten. Jede Story beschreibt eine Funktion, die der Kunde möchte. Eine solche User Story könnte wie folgt lauten:
\begin{quote}
Anzeige einer Druckvorschau beim Auswählen der Druckfunktion
\end{quote}
Danach wird ein Release-Plan erstellt. In der Regel werden alle drei Monate Releases veröffentlicht. Bei der Erstellung des Plans legen die Entwickler fest, welche User Story in welchem Release implementiert werden soll. Der Zeitraum zwischen zwei Releases wird wiederum in ein- bis drei-wöchigen Iterationen unterteilt. Zu Beginn jeder Iteration wählt der Kunde, die für ihn wichtigsten User Stories aus. Die Entwickler entwerfen, implementieren und testen die ausgewählten Stories innerhalb der Iteration. Am Ende jeder Iteration muss das Programm lauffähig sein, auch wenn noch wichtige Funktionen fehlen. 

\subsubsection{Management}
Auch für das Management eines Softwareprojekts sind bei XP klare Regeln definiert. Täglich gibt es ein Meeting, bei dem jeder Entwickler kurz sagt, was er am Vortag getan hat, was er heute tun wird und welche Probleme er sieht. Eine weitere wichtige Regel ist, dass ein Entwickler nicht länger arbeiten darf, als er auf unbefristete Zeit leisten kann. Normalerweise sind dies acht Stunden pro Tag. Überstunden sind nicht gern gesehen. Des Weiteren soll jeder Entwickler in den gesamten Prozess eingebunden sein und immer über den Gesamtstand der Projektes informiert sein. Außerdem sollen Entwickler immer wieder mit anderen Aufgaben betraut werden. Damit soll vermieden werden, dass sie den Anschluss zum technologischen Geschehen nicht verlieren. Aus dem selben Grund soll den Entwicklern auch genügend Freiraum gegeben werden, um auch während eines Projekts sich mit technischen Neuerungen zu befassen.

\subsubsection{Entwurf}

\subsubsection{Testmanagement}

\subsubsection{Programmierung}

\subsection{Scrum}

\subsection{Software Kaban}


\section{Einleitung}
\label{sec:einfuehrung}
Bis zur Jahrtausendwende waren \emph{V-Modell} und weitere Vorgehensmodelle, die auf Requirements Engineering aufbauen, vorherrschend in der Softwareentwicklung. Doch mit dem 1996 veröffentlichen \emph{Extreme Programming} änderte sich die Art der Planung, Entwicklung und Wartung von Software grundlegend. Dieser Prozess ist bis heute noch nicht abgeschlossen, jedoch gelten laut einer Studie von \emph{Forrester Research} agile Softwareentwicklungsprozesse inzwischen als ``Mainstream''. \cite{bib:ane}

Traditionelle Vorgehensmodelle gelten als sehr dokumentationslastig und schwergewichtig. Die Mitte der 90er Jahre entwickelten Vorgehensmodelle, die die Dokumentation eines Programms als zweitrangig ansahen, wurden daher als leichtgewichtig bezeichnet. Erst mit dem \emph{Manifesto for Agile Software Development} \cite{bib:manifest} entstand der Name ``Agile Vorgehensmodelle''. \cite{bib:eckstein} Das vor zehn Jahren veröffentliche Agile Manifest formuliert vier Wertvorstellungen, die für agile Projekte gelten sollen:
\begin{enumerate}
	\item Individuen und Interaktionen mehr als Prozesse und Werkzeuge
	\item Funktionierende Software mehr als umfassende Dokumentation
	\item Zusammenarbeit mit dem Kunden mehr als Vertragsverhandlung
	\item Reagieren auf Veränderung mehr als das Befolgen eines Plans
\end{enumerate}

Des Weiteren stellten die ursprünglich 17 Unterzeichner des Manifests zwölf Grundprinzipien für agile Softwareentwicklung auf. \cite{bib:eckstein} Auf Grund deren Wich\-tig\-keit in der agilen Softwareentwicklung seien diese hier aufgezählt:
\begin{enumerate}
	\item Unsere höchste Priorität ist es, den Kunden durch frühe und kontinuierliche Auslieferung wertvoller Software zufrieden zu stellen.
	\item Heiße Anforderungsänderungen selbst spät in der Entwicklung willkommen. Agile Prozesse nutzen Veränderungen zum Wettbewerbsvorteil des Kunden.
	\item Liefere funktionierende Software regelmäßig innerhalb weniger Wochen oder Monate und bevorzuge dabei die kürzere Zeitspanne.
	\item Fachexperten und Entwickler müssen während des Projektes täglich zusammenarbeiten.
	\item Errichte Projekte rund um motivierte Individuen. Gib ihnen das Umfeld und die Unterstützung, die sie benötigen und vertraue darauf, dass sie die Aufgabe erledigen.
	\item Die effizienteste und effektivste Methode, Informationen an und innerhalb eines Entwicklungsteam zu übermitteln, ist im Gespräch von Angesicht zu Angesicht.
	\item Funktionierende Software ist das wichtigste Fortschrittsmaß.
	\item Agile Prozesse fördern nachhaltige Entwicklung. Die Auftraggeber, Entwickler und Benutzer sollten ein gleichmäßiges Tempo auf unbegrenzte Zeit halten können.
	\item Ständiges Augenmerk auf technische Exzellenz und und gutes Design fördert Agilität.
	\item Einfachheit -- die Kunst, die Menge nicht getaner Arbeit zu maximieren -- ist essenziell.
	\item Die besten Architekturen, Anforderungen und Entwürfe entstehen durch selbstorganisierte Teams.
	\item In regelmäßigen Abständen reflektiert das Team, wie es effektiver werden kann und passt sein Verhalten entsprechend an.
\end{enumerate}
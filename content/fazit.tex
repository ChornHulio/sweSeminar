\section{Fazit}
Agile Vorgehensmodelle bei der Softwareentwicklung werden zunehmend häufiger eingesetzt und gelten inzwischen als ``Mainstream''. \cite{bib:ane} Insbesondere Scrum und Extreme Programming erhöhen die Flexibilität, Effizienz, Transparenz und Qua\-li\-tät des Softwareprojekts. Aus den Projekterfahrungen der Autoren zeigt sich auch eine deutliche Motivationssteigerung der Beteiligten beim Einsatz agiler Methoden. Dies lässt sich mit der erhöhten Zufriedenheit aus den vorher ge\-nann\-ten Punkten erklären.

Agile Vorgehensmodelle scheitern jedoch, wenn zu wenig Know-How im Entwicklerteam vorhanden ist. Denn schafft es ein Team nicht die grundlegende Architektur der Software modular und erweiterbar zu gestalten, ist eine Weiterentwicklung daran fast unmöglich.

Auch bei zu großer Trägheit der Beteiligten, kann kein direkter Umstieg auf Scrum oder XP erfolgen. Hier kann die Change-Management-Methode Software Kanban eingesetzt werden, um aus der Revolution eine Evolution zu gestalten.

Open Source Projekte harmonieren sehr gut mit einer agilen Softwareentwicklung. Insbesondere liegen hier die Stärken der Methoden in der Transparenz der Entwicklung und der Flexibilität auf Änderungen zu reagieren. Lediglich die räumliche Entfernung der Projektbeteiligten spricht gegen die Ideale einer agilen Softwareentwicklung. Trotzdem ist ein erfolgreiches Arbeiten im Team möglich. Dies zeigen zum Beispiel die Arbeiten an TYPO3 und Firefox. 

Unterstützt werden agile Vorgehensmodelle durch einige Programme. Der Mehrwert einer software-gestützten Projektverwaltung liegt in der Transparenz. Die Anwendungen erlauben, dass jedes Projektmitglied stets den aktuellen Stand des Projektes mit verfolgen kann. Dies erleichtert die Kommunikation zwischen den einzelnen Projektmitgliedern, da alle eine gemeinsame Infor\-mations\-quelle besitzen. Die Zeiterfassung wird bei vielen Projektverwaltungswerkzeugen vereinfacht und eine separate Buchführung wird nicht mehr benötigt. Nahezu jedes Werkzeug besitzt eine Anzeigetafel, die eine übersichtliche Möglichkeit zur Darstellung von Taskzuständen eines Sprints ermöglicht. Dadurch können die Entwickler schnell eine Übersicht der aktuellen Arbeit und noch anstehender Tasks bekommen. Die Auswahl von noch abzuarbeitenden Tasks kann leicht vom Arbeitsplatz erfolgen. Die Webfähigkeit der Anwendungen kann eine standortübergreifende Entwicklung erleichtern.

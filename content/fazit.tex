\section{Fazit}
Wie eine Studie von \emph{Forrester Research} zeigt werden agile Vorgehensmodelle bei der Softwareentwicklung zunehmend häufiger eingesetzt und gelten inzwischen als ``Mainstream''. \cite{bib:ane} Als Paradebeispiele für agiles Vorgehen gelten Extreme Programming und Scrum, welche auch zu den beliebtesten Methoden zählen. Sie geben den Entwicklern und Projektbeteiligten wenige klare Regeln an die Hand und steigern dadurch Flexibilität, Effizienz und Transparenz. \cite[S. 28 f.]{bib:wolfRoock} Wenn die Widerstände gegen eine Umstellung auf eine der Methoden zu groß sind, kann die Change-Management-Methode Software Kanban einen sanften Einstieg in die agile Softwareentwicklung zu ermöglichen. Hier wird eine evolutionäre Umstellung der bestehenden Strukturen durchgeführt.

Open Source Projekte harmonieren sehr gut mit einer agilen Softwareentwicklung. Dies zeigen zum Beispiel die Arbeiten an TYPO3 und Firefox. 

Unterstützt werden agile Vorgehensmodelle durch einige Programme. Diese ermöglichen eine transparente Projektplanung und -durchführung. 
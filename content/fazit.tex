\section{Fazit}
Wie eine Studie von \emph{Forrester Research} zeigt werden agile Vorgehensmodelle bei der Softwareentwicklung zunehmend häufiger eingesetzt und gelten inzwischen als ``Mainstream''. \cite{bib:ane} 

Open Source Projekte harmonieren sehr gut mit einer agilen Softwareentwicklung. Dies zeigen zum Beispiel die Arbeiten an TYPO3 und Firefox. 

Unterstützt werden agile Vorgehensmodelle durch einige Programme. Diese ermöglichen eine transparente Projektplanung und -durchführung.

%noch richtig in den text einbauen...bitte
Der Mehrwert einer software-gestützten Projektverwaltung liegt in der Transparenz. Die Anwendungen erlauben, dass jedes Projektmitglied stets den aktuellen Stand des Projektes mit verfolgen kann. Dies erleichtert die Kommunikation zwischen den einzelnen Projektmitgliedern, da alle eine gemeinsame Informationsquelle besitzen. Die Zeiterfassung wird bei vielen Projektverwaltungswerkzeugen vereinfacht und eine separate Buchführung wird nicht mehr benötigt. Nahezu jedes Werkzeug besitzt eine Anzeigetafel, die eine übersichtliche Möglichkeit zur Darstellung von Taskzuständen eines Sprints ermöglicht. Dadurch können die Entwickler schnell eine Übersicht der aktuellen Arbeit und noch anstehender Tasks bekommen. Die Auswahl von noch abzuarbeitenden Tasks kann leicht vom Arbeitsplatz erfolgen. Die Webfähigkeit der Anwendungen kann eine standortübergreifende Entwicklung erleichtern.

\section{Projektverwaltungswerkzeuge}
(Autor: Vogt)\\

\subsection{Anforderungen}
Die Anforderungen an ein Projektverwaltungswerkzeug unterscheiden sich je nach Projektkomplexität und Anzahl der Teilnehmer. Grundsätzlich gilt, je größer die Anzahl der Teilnehmer, desto ratsamer ist eine software-gestützte Verwaltung einzusetzen.
Das Projektverwaltungswerkzeug für die aglie Softwareentwicklung muss dabei die gängigen agilen Projektverwaltungsaufgaben unterstützen. Diese funktionalen Anforderungen sehen wie folgt aus:

\begin{description}
	\item[Ressourcenverwaltung]\hspace*{1em}\\
Die Ressourcenverwaltung des Projekts erlaubt die Überwachung und Planung von Ressourcen wie die Kapazität von Mitarbeitern, das Budget und den Zeitaufwand.
	\item[Rollenverteilung]\hspace*{1em}\\
Die Rollenverteilung bestimmt die Rolle jedes Mitarbeiters in einem Projekt.
	\item[Zeitverwaltung]\hspace*{1em}\\
Diese Anforderung erlaubt die Analyse von Zeitaufwänden von Stories/Tasks.
	\item[Aufgabenverteilung]\hspace*{1em}\\
Die Aufgabenverteilung regelt die Auswahl der zu bearbeitenden Stories/Tasks, die von den Entwicklern eigenständig ausgewählt werden können.
	\item[Problemmanagement]\hspace*{1em}\\
Das Problemmanagement dient zum Erfassen von Korrekturvorschlägen und die Meldung von Fehlverhalten in bereits erstellter Software. 
	\item[Releaseplanug]\hspace*{1em}\\
Die Vergabe von festen Termine für den Release von bestimmten Softwareständen zu planen.
\end{itemize}

Desweiteren unterliegt das Projektverwaltungswerkzeug allgemeinen Software Anforderungen. Mit diesen nicht-funktionalen Anforderungen lässt sich das passende Produkt für den Produktiveinsatz ermitteln.
\begin{description}
	\item[Benutzbarkeit]\hspace*{1em}\\
Die Benutzbarkeiit ist die Bedienfreundichkeit einer Anwendung. Diese Anforderung zeigt, wie leicht man eine graphische Oberfläche bedienen kann, wie übersichtlich und wie intuitiv sich diese Bedienen lässt.
	\item[Effizienz]\hspace*{1em}\\
Die Effizienz zeigt an wie hoch der Ressourcenverbrauch der Software ist. Zu den wichtigen Ressourcen gehören der Speicher- und die CPU-Auslasung.
	\item[Wartbarkeit]\hspace*{1em}\\
Die Anforderung der Wartbarkeit beschriebt die Möglichkeit das System durch beispielsweise Korrekturen oder Anpassungen zu ändern.
	\item[Portierbarkeit]\hspace*{1em}\\
Die Portierbarkeit einer Software sagt aus, wie leicht sich eine Software auf andere Systeme portieren lässt, ob andere Systeme unterstützt werden oder die Software austauschbar ist.
	\item[Zuverlässigkeit]\hspace*{1em}\\
Die Zuverlässigkeit der Software zeigt die Fehlertoleranz gegenüber Eingaben, das Wiederherstellen bei Abstürzen und eine konsistente Datenhaltun. 
	\item[Lizenzierung]\hspace*{1em}\\
Diese Anforderung bezieht sich auf die rechtliche Verwendungslage von Software. Darin werden der Einsatz und die Verbreitung der Software geregelt.
	\item[Skalierbarkeit]\hspace*{1em}\\
Die Skalierbarkeit bezieht sich auf die Anzahl der Benutzer und der Anzahl der zu verwaltenden Projekte.
\end{description}

\subsection{Mehrwert}
Der Mehrwert einer software-gestützten Projektverwaltung \dots 

\subsection{Propriet"are Software}

\subsection{Open-Source Software}


\begin{abstract}
Diese Seminararbeit beschäftigt sich mit agilen Methoden der Softwareentwicklung in Bezugnahme auf die Verwendung in Open-Source Projekten. Dabei werden die Entstehung der agilen Vorgehensweisen und die Methoden von \emph{Extreme Programming, Scrum} und \emph{Software Kanban} genauer erläutert. Danach wird die Verwendung von agiler Vorgehensweise bei größeren Open-Source Projekten wie beispielsweise \emph{TYPO 3} genauer betrachtet. Zum Schluss werden mehrere Projektverwaltungswerkzeuge vorgestellt, die speziell auf die agile Entwicklung zugeschnitten sind, den Verwaltungsaufwand verringern und die Produktivität steigern sollen.
\end{abstract}
